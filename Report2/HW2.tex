\documentclass[a4paper,10pt]{report}
\usepackage[utf8]{inputenc}

% Title Page
\title{ME608 Homework 2}
\author{Shouyuan Huang}

\usepackage{amsmath}
\usepackage{graphicx}
\begin{document}
\maketitle
\section{Problem 1 TDMA and Line by line TDMA}


For this application, TDMA and TDMA lbl are realized by MATLAB script, in TDMA.m and lblTDMA.m. 

From the falsifiability, we can never verify the correctness of a code by testing it. However, two piece of testing script
are attached to show that the solver at least does the specific problems right. A rank 5 linear algebraic equation is made
up and comparison is made between TDMA and inv() function provided by MABLAB. A 2D heat conduction problem is set up to 
test the lblTDMA, the solution is actually compared with the result in the heat transfer textbook by 
Incroppera.\footnote{Incropera, Frank P. Fundamentals of heat and mass transfer. John Wiley \& Sons, 2011.}
The validity is also "varified" through Problem 2. \footnote{Problem 2 is a 2D Poiseuille flow, whose profile can be 
easily recognized. }

The corresponding script: \emph{TDMA.m}, \emph{lblTDMA.m}, \emph{testTDMA.m}, \emph{testlblTDMA.m} are attached in Appendix.

%The following piece of script is to insert a equation.
%\newtheorem{eqn}{Equation}
%\begin{eqn}[Diffusive Law]
%\begin{equation}
% \div^2 u = dp/dt
%\end{equation}
%\end{eqn}


%\begin{figure}[ht]
%ht: h indicates here, t indicates at the top of a page.
%  \centering
%  \includegraphics[option of size]{filename of figure}
%  \caption{the caption here}
%  \label{fig:filename}
%\end{figure}
\section{Problem 2 2D Poiseuille flow in rectangular annulus}


The problem is implemented in MABLAB script \emph{HW2P2.m}. The solution region is chosen at the northeast quarter of the 
crossaction. As a result of the symmetric separation, the boundary condition across the annulus is zero Neuman type.  
This script allows to set size of the duct and thickness of the annulus. On mesh, it allows to set cell number both along 
and across the annulus. Different mesh sizes on x and y directionare allowed, the interface is treated accordingly, but 
smooth transition is not implemented. \footnote{This feature is not used, though implemented, because I have no idea 
how to optimize the grid size distribution to improve the result. }


\begin{figure}[ht]
\centering
\includegraphics[width=5cm]{P2_50_C.png}
\caption{The velocity distribution at the centerline of coarse($50\times50$) mesh.}
\label{P2CtrC:P2_50_C.png}
%\end{figure}
%\begin{figure}[ht]
%\centering
\includegraphics[width=5cm]{P2_200_ctr.png}
\caption{The velocity distribution at the centerline of fine($200 \times 200$) mesh.}
\label{P2CtrF:P2_200_ctr.png}
\end{figure}


Due to the limitation of computational resource, the mesh independency is not closely, but not completely reached. 
The maximun grid number tried is $(80+120) \times (80+120)$. Plot at the center line (Fig \ref{P2CtrC:P2_50_C.png} and 
\ref{P2CtrF:P2_200_ctr.png}), at the annulus center, as well as the mesh of overall velocity distribution are shown in appendix.
trend is shown clearly with the plot and mesh. 
For coarse mesh, a very noticable phenomena is that the mesh fineness highly affects how much the Dirichlet Boundary 
Condition limits its neighbor elements. Or say, the more the mesh refined, the closer the result is to the preset 
no slip condition.


\section{Problem 3 Reaction Flow within duct annulus}


For \emph{fully developed} flow, along the flow, $ \frac{\partial \theta}{\partial x} = constant $ ,in which $ \theta $ 
is nondimensionalized temperature. For Neuman boundary all around, $ \theta $ is defined as: $ \theta = \frac{T - T_b}{T_b}$, 
where $T_b$ is bulk temperature. Otherwise, $ \theta = \frac{T - T_b}{T_{ref} - T_b} $. This leads to $ \frac{\partial T}{\partial x} = constant $. 


From the governing equation, i.e. Energy Eqn:
\begin{equation}
 \rho C_p \frac{\partial T}{\partial x} u = \nabla \cdot ( k \nabla T ) + \dot q
\end{equation}
When taken integral with y-z plane, the Laplacian term eliminates due to its zero-Neuman boundary condition. Thus, the result 
shows bellow:
\begin{equation}
 \frac{\partial T}{\partial x} = \frac{\int_A \dot q dA}{\rho C_p \int_A u dA} = \frac{\overline{\dot q}}{\rho C_p \overline{u}}
\end{equation}

The temperature field is discretized and solved by the MATLAB script HW2P3.m. 
The program with proper under-relaxation ratio will natually converge at a certain bulk temperature, but depends on the 
choice of $\alpha$ and $\epsilon$ (convergence creterion), which is uncontrollable.A special treatment is 
applied to specify the location of the crossaction. The temperature at the northeast corner is 
set constant at $ T = 100 ^\circ C $. Of course, this will not affect the result showned, which is nondimensionalized. 
Also, the script is required to run after HW2P2.m. This is because it is using the mesh generated in and the calculation 
result of velocity field from the latter script. 


The nondimensionalized temperature profile at the horizontal central line is shown a follows:
\begin{figure}
 \centering
 \includegraphics[width = 5cm]{P3.png}
 \caption{Nondimensionalized temperature at horizontal  line}
\end{figure}

\section{Problem 4 Discrete Equation for Radiation}


Clearly, the radiative term should be treated as a source term. This is to say that $a_{nb}$ should be unchanged. And the 
radiative source should be linearized, after which process, the constant term goes to $b$, the $1^st$ order term goes to
$a_p$. 


For uniform grid, $ \Delta x = \Delta y = \delta x = \delta y \overset{\Delta}{=} \Delta $ Thus,
\begin{equation}
 a_W = k;
 a_E = k;
 a_S = k;
 a_N = k;
\end{equation}
To solve the radiation problem, we actually require iterative method with linearation of the radiative source term. 


Define $T*$ as the temperature at p in the previous iteration, and for the first, the initial guess. then define: 
$\Delta T = T - T* $. 
Then by Taylor expansion($2^nd$ order terms neglected):

\begin{equation}
\begin{split}
 \overline S &= S_C + S_P T \\
             &= \kappa G - \kappa \sigma (T* + \Delta T)^4 \\
             &= (\kappa G - \kappa \sigma T*^4 - 4 \kappa \sigma T*^3 \Delta T)\\
             &= (\underbrace{\kappa G + 3\kappa \sigma T*^4}_{S_C}) + \underbrace{-4 \kappa \sigma T*^3}_{S_P} T 
\end{split}
\end{equation}



i.e. 
\begin{equation}
 a_P = a_E + a_W + a_N + a_S - S_P\Delta x \Delta y = 4k + 4\kappa \sigma T*^3 \Delta^2 
\end{equation}
\begin{equation}
 b = S_C \Delta x \Delta y 
    = (\kappa G + 3\kappa \sigma T*^4)\Delta^2
\end{equation}


Linearization \& Disretization Done! 


\end{document}          


